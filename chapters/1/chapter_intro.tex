\startfirstchapter{Introduction}
\label{chapter:introduction}

In 2020, of 10.35 million new cases of cancer in low and middle-income countries (LMIC) more than 70\% had no access to proper treatment for their disease\cite{CancerCountries/}. The development of low-cost radiotherapy treatment machines is imperative for improving cancer outcomes for patients in LMIC as x-ray treatment is effective as a single or combined modality for local–regional tumor control \cite{UN2012,WorldhealthOrganization2018WHODiseases}. To treat the unserviced population in LMICs the World Health Organisation estimates 5000 megavoltage (MV) radiotherapy linear accelerators (linacs) are needed\cite{Samiei2013ChallengesCountries}. With new linacs costing on the order of \$5M USD, the sort of investment for procurement and operation of this many machines will likely not be garnered by LMICs. This lack of low cost treatment options in LMICs is the premise of my PhD work, and specifically, I am investigating two discrete solutions to lowering the cost of these radiotherapy machines: The first is through the elimination of the costly kilovoltage on-board imagers (kV-OBI) through the introduction of novel MV imaging detector materials in combination with low atomic number linac targets while the second is the development of a super low cost kilovoltage radiotherapy system in collaboration with California based startup Sirius Medicine LLC.

To motivate the necessity to improve MV imaging technology it is helpful to have some background in recent radiotherapy development: Currently, there is a huge potential for increasing the throughput of conventional radiotherapy machines. With new stereotactic ablative radiotherapy (SABR) treatments being used throughout the developed world\cite{ClaridgeMackonis2020StereotacticNSW,Palma2019StereotacticTrial,Palma2020StereotacticTrial,Billiet2020OutcomeOligometastases,Li2020RadiobiologyOncologists}, cancer sites that once would have required a month of once per day radiotherapy treatment can now be treated within a week using very high doses of radiation. This opens up the possibility of very high patient throughput for some treatment sites if these treatments can be performed in LMICs\cite{Asher2019PracticalWorld}. 

However, to perform these SABR treatments some technical aspects must be addressed that prove challenging in LMICs. A key challenge in SABR treatment is to position the patient accurately during treatment so that the treatment beam precisely hits the patient's tumour\cite{Mesko2020EstimatingAccuracy,Wang2016ImprovedCancer}. The best way to ensure proper treatment positioning is to use a kilovoltage cone beam CT system (kV-CBCT) mounted to the treatment machine. Unfortunately, these kV-CBCT systems contribute significantly to the cost of the radiotherapy machine: While a pre-owned treatment machine without a kV-CBCT system can cost as low as \$175,000 USD, pre-owned machines with kV-CBCT capability are generally over three times this amount\cite{2021Linear/resources/linear-accelerator-guides/used-linac-price}.

\subsection{MV CBCT Improvement}

To avoid the added cost of a kV-CBCT system, megavoltage CBCT (MV-CBCT) can be used to acquire CT scans with the treatment beam itself\cite{Held2015FeasibilityCalculations,Malajovich2019CharacterizationDoses}. The downside of MV-CBCT is that the CT images acquired have lower contrast and require a higher patient radiation dose to maintain equivalent image quality to kV-CBCT\cite{Star-Lack2015AImaging}. Current solutions for improving the image quality of MV-CBCT include introducing multiple beam modes, some with characteristics for treatment and others with characteristics for imaging\cite{Parsons2012BeamTargets,Parsons2013PlanarLinear,Robar2012Volume-of-interestTarget}. Likewise, the x-ray detector design is an active area of development with new detector materials providing higher detection efficiency of x-rays, resulting in lower noise for MV beams\cite{Star-Lack2015AImaging,Myronakis2020Low-doseMLI,Myronakis2017AOptimization:}. 
To improve MV imaging quality, novel low atomic number (Z) beam targets have been examined to generate photon beams with higher fluence in the kV range. The use of aluminum and carbon targets has been shown to improve image quality: Simulation studies show a 16-19\% increase in contrast between an aluminum target and the default tungsten target for a 6 MV photon beam \cite{Kim2018InvestigationSimulation, Flampouri2002OptimizationExperiment}. Experimentally, Baek et al. showed that a gantry mounted aluminum target improved the limiting spatial frequency (f50) from 0.451 lp/mm to 0.745 lp/mm for a tungsten target  6 MV beam\cite{Baek2019AssessmentFilter}. Additionally, Parsons et al. showed an increase in the contrast to noise ratio (CNR) in projection images of cortical bone by factors ranging from 3.7 to 7.4 between carbon and aluminum targets with 2.35 and 1.9 MV photon beams and the default 6 MV tungsten imaging setup \cite{Parsons2012BeamTargets}.

Furthermore, advances in MV EPID design have shown promise in improving MV image quality. Many MV EPIDs use a gadolinium oxysulfide (GOS) scintillator, which is opaque to its own scintillation photons. This limits the thickness of GOS detectors which in turn limits the quantum efficiency of the detector. Star-lack \textit{et al.} examined cadmium tungstate (CWO) and bismuth germinate (BGO) detectors, which are higher Z materials and can be made thicker for MV photon detection. CWO was noted to have a 20-fold efficiency improvement over GOS with significantly higher stability and light yield than BGO \cite{Star-Lack2015AImaging}. Likewise, image quality improvement was seen using multi-layer GOS imagers with 2-4 times greater CNR than an equivalent single layer GOS detector \cite{Myronakis2020Low-doseMLI}.

An additional novel and cheap material that has very recently been used for medical imaging applications and is well suited for MV imaging is Perovskite crystals. These crystals which see application in solar cell production have recently demonstrated excellent spatial resolution for kV imaging\cite{Gill2018FlexibleApplications,Butey2021CurrentTechnology,Zhu2020Low-doseScintillators}. Their lead content, high spatial resolution, and cheap production costs make them an attractive candidate as an MV detector material. However, the use of Perovskite crystals in MV imaging is currently an unexplored topic.

Currently, these two approaches to improving MV imaging, the amelioration of detector efficiencies and introduction of low-Z targets, have remained largely separate. Likewise, direct comparison of the benefit of different imaging strategies in the literature remains challenging as different works have different combinations of voxel size, cone-beam size, phantom dose, focal-spot size and other imaging variables.

\subsubsection{Improved CBCT Simulation}

To adequately quantify the quality of these novel MV-CBCT imaging setup Monte Carlo simulation codes are generally used. However, while MC simulations are accurate they are exceptionally computationally demanding. A large number of particles need to be simulated to form a CBCT image, especially when simulating the special detector materials in medical imaging, called scintillators, which convert x-ray photons to many optical photons which can be measured using typical optical readout methods. The long simulation time can be primarily attributed to two factors. First, MV detectors generally have low detective quantum efficiency (DQE), meaning that many particles that are transported in the simulation do not interact with the detector and do not contribute to image formation. A typical EPID DQE is as low as 1\%-1.5\% \cite{Myronakis2017AOptimization:,Hu2017APerformance}. Second, the scintillating detector in which the optical photons are produced generally has a high scintillation yield; generating thousands of optical photons to be transported per interaction event. These two factors result in simulation times often as long as 3,000 core-hours for a 10$^7$ primary x-ray simulation of one EPID projection \cite{Blake2013CharacterizationGeant4}. Further, to produce a clinically equivalent image of 1 MU with a 10$\times$10cm$^2$ field size a simulation with more than 10$^{11}$ photons is required\cite{Shi2019ADetectors.,Star-Lack2014RapidDQEf,Rottmann2016AEfficiency}. This problem is compounded in CBCT simulation, where it is necessary to simulate many projections of the object for CBCT image reconstruction.

Large headway has been made to reduce this computational overhead. Star-lack \textit{et al.}  have shown that one can simulate the detector response with only a fraction of the scintillation yield, significantly reducing the computation time\cite{Star-Lack2014RapidDQEf}. Likewise, by simulating the optical spread function at discrete energies beforehand and convolving these optical spread functions with the energy deposition of an absorbed photon one can avoid simulating the scintillation processes completely\cite{Kausch1999MonteRadiotherapy, Kirkby2005ComprehensiveEPID}. Additionally, Shi \textit{et al.} introduced the fastEPID framework which pre-calculates energy deposition efficiency ($\eta$) and optical spread function (OSF) to remove particle transport in the detector entirely without loss of image quality \cite{Shi2019ADetectors.}. Where $\eta$ is defined as the ratio of the total energy deposition in the scintillator and the total x-ray photon energy incident on the detector. However, these simulations are still considered computationally intensive with one image at 1 MU taking 1.540 $\times$ 10$^4$ core-hours on an Intel Skylake CPU core (Intel Corp., Santa Clara, Ca). Other MC approaches that are promising for image simulation are the GPU methods used by Badal and Badano as well as Bert \textit{et al.}\cite{Bert2013Geant4-basedApplications,Badal2009AcceleratingUnit}. The approaches have seen MC simulation speedup factors of 27 and 80-90, respectively. However, at this time, open source GPU MC codes lack the stability and versatility of more established codes such as Geant4 or EGSnrc.

Additionally, a number of works simulate fan and cone beam CT analytically to reduce computation time. ImaSim analytically simulates fan and cone beam CT through raytracing using vectorized phantoms \cite{Landry2013ImaSimRadiology}. VOXSI simulates kV fan beam CT using analytical raytracing with voxelized phantoms \cite{vanderHeyden2018VOXSI:Imaging} and shows agreement with experimental images in terms of image contrast. DukeSim simulates kV fan beam CT with voxelized phantoms through a combination of analytical raytracing and GPU MC and demonstrates agreement between experimental and simulated images in terms of image contrast, noise magnitude, noise texture, and spatial resolution \cite{Abadi2019DukeSim:Tomography}. A drawback to the DukeSim approach is that the GPU MC reduces the simulation speed, requiring a 2-3 minute simulation per source rotation while running on 4 Nvidia Titan Xp GPUs with 64 GB of memory. Overall, none of these platforms are open-source and only DukeSim shows agreement with experimental noise and image contrast. Additionally, none of these platforms show experimental agreement for kV or MV CBCT.

In my PhD work I used a modified fastEPID pre-calculation of MC data for the detector response while also pre-calculating the energy spectrum of the beam source as well as energy dependent scatter kernels for a cylindrical water phantom. I combined this data with an analytical GPU raytracer that provides the primary particle attenuation. This simulation strategy is used to create Fastcat, an open source simulation tool for CBCT image simulation to enable studies of novel beam, detector, and phantom combinations. Fastcat shows good agreement with MC simulations of full CBCT data acquisition and it results in extremely short run times on the order of 1 GPU-minutes for a full CBCT simulation.

In my PhD work I also simulate image quality for combinations of novel MV beam target materials such as carbon and aluminum with different detector materials such as CWO and a novel Perovskite crystal detector. I compare these novel imaging methodologies with standard kV imaging setup devices such as columnar cesium iodide (CsI) detectors. I investigated whether the combination of novel MV beams and detectors can result in MV CBCT image quality approaching kV CBCT image quality.

\subsection{kV Treatment Machine Design}

Conversely, to decrease the cost of radiotherapy by an order of magnitude one can bypass linacs and use kV treatments machines. While kV treatment machines cannot be expected to have comparable treatment efficacy and versatility to MV linear accelerators, their low cost makes them the only approach realistic for some LMIC. Modern radiotherapy treatment machine design centres around the use of a linear accelerator to create MV photon beams. These beams are ideally suited to this task in that they can deposit cell damaging radiation in cancerous lesions deep in the human body while sparing the sensitive skin from radiation burns. 

Historically, throughout the early 20th century, kV photon beams were used in this application since they are simple to produce and have the same cell damaging properties as MV x-rays, however, these kV photon beams generally produced radiation burns on the patients’ skin. Currently, with advances in x-ray tube development, treatment planning optimization, and robotics there are opportunities to re-introduce kV photon beams as a treatment option for deep lesions in LMICs where treatment options are currently unavailable\cite{Breitkreutz2020ExternalX-rays}. Additionally, kV photon beams show major benefits when used in combination with gold nanoparticles (GNPs) as a radiosensitizer \cite{Sung2018EnergyTherapy}. An application in which there is currently no commercial solution in place to treat humans with a kV treatment machine designed to treat deep lesions.

A key design feature of a kV treatment machine is to have multiple beams converging on the tumour. This allows the skin dose to be spread out over a large area allowing there to be sufficient dose delivered to the tumour to sufficiently treat the disease while no individual area of skin is given enough of a radiation dose to cause painful side effects for the patient. An additional attractive feature of kV x-rays incident on a high atomic number (Z) material such as gold or iodine will generally interact using the photo-electric effect. The photo-electric effect in turn produces Meitner electrons, sometimes referred to as Auger-electrons, these electrons deposit dose locally and are effective tumor radiosensitizers, increasing the efficacy of kV radiotherapy treatments. Previous designs have used a scanning electron beam over many targets to create many converging beams \cite{Breitkreutz2017MontePatients:}.

In 1999, a kV treatment modality constructed out of a repurposed CT scanner, termed CTRx, used a modified pencil-beam collimator to treat brain lesions in clinical trials  \cite{Rose1999FirstCTRx, Mesa1999DoseAgents}. CTRx was capable of IGRT due to its ability to function as both a diagnostic and therapeutic system. Mesa \textit{et al.} additionally used Monte Carlo dose calculations to investigate the dosimetric properties of the CTRx system used in conjunction with Iodine as a radiosensitizer, a topic that will be discussed further below. The energy spectrum used in the simulations was modelled after the 140 kVp setting of a GE CT scanner. Skull dose reduction was explored both by means of increasing the concentration of iodine (up to 20 mg/ml) in the tumor as well as the use of three non-coplanar (20$^{\circ}$, 0$^{\circ}$ and -20$^{\circ}$) treatment arcs by tilting the gantry. The CTRx treatments were compared to simulated treatments with a conventional 10 MV photon beam modelled after a linac. The CTRx system was capable of achieving suitable tumor dose distributions given sufficient iodine concentrations and non-coplanar beams.

In my work I explore a novel design for a kV treatment system using a conventional x-ray tube mounted on an articulating robotic arm. This system leverages the decreased cost of robotic arms which for prices around \$30K USD can handle the weight of an x-ray tube and have sub millimeter positioning accuracy. In combination with a commercial 320 kV industrial x-ray tube these arms could provide a non-coplanar kV-treatment machine for less than \$150K USD, with low operating costs, few specialised components, and low shielding requirements. These treatment machines, although likely not as effective as MV treatment machines, would avoid the estimated \$5M USD price of a new linac and \$2M USD additional cost of constructing the 2 metre thick cement bunker.
